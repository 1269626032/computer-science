\documentclass{article}
\usepackage{ctex}
\usepackage{hyperref}
\usepackage{longtable}
\usepackage{booktabs}

\newcommand{\tightlist}{
  \setlength{\itemsep}{0pt}
  \setlength{\parskip}{0pt}
}

\title{春松客服项目申请书}
\date{}
\usepackage{listings}
\lstset{
    % backgroundcolor=\color{red!50!green!50!blue!50},%代码块背景色为浅灰色
    rulesepcolor= \color{gray}, %代码块边框颜色
    breaklines=true,  %代码过长则换行
    numbers=left, %行号在左侧显示
    numberstyle= \small,%行号字体
    keywordstyle= \color{blue},%关键字颜色
    commentstyle=\color{gray}, %注释颜色
    frame=shadowbox%用方框框住代码块
    }



\begin{document}
\maketitle
\hypertarget{ux9879ux76eeux540dux79f0}{%
\section{项目名称}\label{ux9879ux76eeux540dux79f0}}

项目二:春松客服的接口开发

\hypertarget{ux9879ux76eeux8be6ux7ec6ux65b9ux6cd5}{%
\section{项目详细方法}\label{ux9879ux76eeux8be6ux7ec6ux65b9ux6cd5}}

\hypertarget{ux8f6fux4ef6ux4f9dux8d56}{%
\subsection{软件依赖}\label{ux8f6fux4ef6ux4f9dux8d56}}

\begin{enumerate}
\item
  操作系统

  Ubuntu 16.04+
\item
  Docker

  Docker
  是一个开源的应用容器引擎,让开发者可以打包他们的应用以及依赖包到一个可移植的镜像中,然后发布到任何流行的
  Linux或Windows
  机器上,也可以实现虚拟化。容器是完全使用沙箱机制,相互之间不会有任何接口。
  \textgreater{} 安装Docker: \textgreater{}
  文档:https://docs.docker.com/engine/install/ubuntu/
  安装成功之后执行以下两个命令

  \begin{longtable}[]{@{}ll@{}}
  \toprule
  \endhead
  命令 & 含义\tabularnewline
  \hline
  sudo groupadd docker & 为Docker创建group\tabularnewline
  sudo usermod -aG docker \$USER &
  将登录者加入到Docker的group里\tabularnewline
  \bottomrule
  \end{longtable}
\item
  Docker Compose

  定义和运行多个 Docker 容器的应用 Docker Compose
  是一个工具,这个工具可以通过yml文件定义多容器的Docker应用
  通过一条命令就可以根据yml文件的定义去创建或者管理这多个容器。
  安装Docker Compose : 文档:https://docs.docker.com/compose/install/
\item
  Git

  安装Git以及常用工具

  \begin{longtable}[]{@{}ll@{}}
  \toprule
  \endhead
  命令 & 含义\tabularnewline
  \hline
  sudo apt-get update & 更新\tabularnewline
  sudo apt-get install wget git curl vim -y &
  安装一些常用工具\tabularnewline
  \bottomrule
  \end{longtable}
\item
  Java JDK-Amazon Corretto

  \begin{enumerate}
  \item
    安装Java JDK-Amazon Corretto

    文档:https://aws.amazon.com/cn/corretto/
  \item
    设置环境变量

    \begin{longtable}[]{@{}ll@{}}
    \toprule
    \endhead
    名称 & 值\tabularnewline
    \hline
    export JAVA HOME &
    /usr/lib/jvm/java-1.8.0-amazon-corretto\tabularnewline
    export PATH & \$PATH:\$JAVA HOME/bin\tabularnewline
    \bottomrule
    \end{longtable}
  \end{enumerate}
\item
  Maven

  \begin{enumerate}
  \item
    安装Maven

    文档:https://maven.apache.org/install.html
  \item
    设置环境变量
  \end{enumerate}
\end{enumerate}

\hypertarget{ux6e90ux7801ux6784ux5efa}{%
\subsection{源码构建}\label{ux6e90ux7801ux6784ux5efa}}

\begin{enumerate}
\item
  官方代码库及镜像

  \begin{longtable}[]{@{}ll@{}}
  \toprule
  名称 & 地址\tabularnewline
  \midrule
  \endhead
  Github & \url{https://github.com/chatopera/cosin}\tabularnewline
  Dockerhub &
  \url{https://hub.docker.com/r/chatopera/contact-center}\tabularnewline
  \bottomrule
  \end{longtable}
\item
  下载源码

  使用Git
\begin{lstlisting}[language={bash}]
  Clone git clone \url{https://github.com/chatopera/cosin.git}
\end{lstlisting}

或者
\begin{lstlisting}[language={bash}]
git clone git@github.com:chatopera/cosin.git
\end{lstlisting}

\item
  文件目录

  \begin{longtable}[]{@{}ll@{}}
  \toprule
  文件 & 含义\tabularnewline
  \midrule
  \endhead
   /cosin & 根目录\tabularnewline
   /cosin/contact-center & 核心的java应用\tabularnewline
   /cosin/contact-center/app & java和前端\tabularnewline
   /cosin/public/plugins & 开源版本的插件\tabularnewline
   /cosin/public/plugins/chatbot &
  机器人客服插件\tabularnewline
  \bottomrule
  \end{longtable}
\item
  安装机器人客服插件
\begin{lstlisting}[language={bash}]
  cd  /cosin
  ./public/plugins/chatbot/scripts/install.sh
\end{lstlisting}

  注意:如果是windows系统的话,需要使用Git
  Bash等Windows下的命令行工具来运行安装脚本
\item
  构建镜像

  \begin{enumerate}
  \item
    生成J2EE应用包
\begin{lstlisting}[language={bash}]
    cd  /cosin/contact-center ./admin/package.sh
\end{lstlisting}

  \item
    执行完之后查看是否生成
\begin{lstlisting}[language={bash}]
    ls ./app/target/*.war
\end{lstlisting}

  \item
    生成Docker镜像

    \begin{enumerate}
    \tightlist
  \item
    \begin{lstlisting}[language={bash}]
            cd  /cosin/contact-center
\end{lstlisting}

\item
  \begin{lstlisting}[language={bash}]
PACKAGEVERSION='git rev-parse --short HEAD'
\end{lstlisting}

    \item
      docker build
    \end{enumerate}
  \end{enumerate}
\item
  成功之后会看到
  build成功
\item
  发布镜像

\end{enumerate}

\hypertarget{ux914dux7f6eux53caux8fd0ux884c}{%
\subsection{配置及运行}\label{ux914dux7f6eux53caux8fd0ux884c}}



  在项目启动完成之后,使用浏览器访问 \url{http://localhost:8035}
  (CC WEBPORT没有变更的情况下默认为 8035)
  默认管理员账号: admin 密码: admin1234

\hypertarget{ux9879ux76eeux5f00ux53d1ux65f6ux95f4ux8ba1ux5212}{%
\section{项目开发时间计划}\label{ux9879ux76eeux5f00ux53d1ux65f6ux95f4ux8ba1ux5212}}

\hypertarget{ux6708}{%
\subsection{7月}\label{ux6708}}

\begin{enumerate}
\item
  7.1-7.4

  春松客服的部署上线
\item
  7.5-7.11

  支持http访问和数据传输
\item
  7.12-7.18

  开发IOS接口
\item
  7.19-7.25

  开发Android接口
\item
  7.20-7.31

  开发JS端口
\end{enumerate}

\hypertarget{section-6}{%
\subsection{8.1-8.31}\label{section-6}}

开发RESTfulAPI

\hypertarget{section-7}{%
\subsection{9.1-9.30}\label{section-7}}
整理代码和修改bug,等待提交
\end{document}
